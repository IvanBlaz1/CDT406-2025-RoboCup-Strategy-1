\section{Introduction}
\label{section:intro}

The RoboCup \cite{RoboCupSSL} is a tournament where different teams 
compete against each other with soccer-playing robots. The RoboCup 
Federation arranges several types of leagues where every league uses 
different types of robots in different shapes and sizes. Overall, this 
tournament aims to advance in the scientific field of mobile robots. 
This project will focus on the Small Size League (SSL), division B in 
particular. This specific league uses a centralised vision system that 
gives allows all robots to get information about the position of the 
other robots and the ball at all times. In that way, developers can focus 
all efforts on the strategical side of the game which makes the SSL 
perfect for newcomers in the RoboCup. In the SSL division B teams 
compete in mathces with 6 against 6 robots and the matches consist of 
two halves where each half is five minutes long with a five-minute pause 
in between. The robots are constrained to certain physical dimensions 
according to the rules (the robots need to fit inside a cylinder of 
0.18 meters in width and 0.15 meters in height) and the robots are built 
by the members of each team. The playing field is 10.4 by 7.4 meters 
with a playing area of 9 by 6 meters and the game is played with an 
orange golf ball. The rules of this league are similar to regular soccer 
but with several modifications. For example the rules include yellow 
and red cards, free kicks and penalties but also rules like maximum 
shooting speed and maximum dribbling length. 

The aim of this project is to develop a system that works well in 
simulation. That will be done by creating an AI system that can 
coordinate all six robots, handle the ball, score goals and defend 
against the opponents. In the long term the models we develop could 
be further developed and used in other works related to both RoboCup 
and other areas. In this paper we aim to answer the question ADD RQ's HERE.
