\begin{abstract}
The RoboCup is a selection of leagues where teams play against eachother with football-playing robots. 
This tournament is an initative taken to advance the scientific field of AI and robotics. For this paper, 
two methods will be explored with the aim to create a working strategy and each method will be tested 
using its own simulation environment. One of the methods used was a proximal policy optimization (PPO) 
which is a form of reinforcement learning (RL) and the goal was to train agents with this method in 
multiple scenarios. In the other method used, a behaviour tree (BT) combined with genetic algorithm (GA) 
was created. The BT controlled what low-level skills the agent used and the GA was used to tune the 
parameters of those skills. This expetiment showed that BT together with GA gave a slightly better 
result than PPO. Unfortunatly, a lot of time went to configuring and perfecting simulation environments 
which led to less time working on the actual training and AI models which in turn led to worse results 
than anticipated. For the future of this experiment, further refinement of the simulation environments 
is required and it would also be suitable to test other AI models to find the most sufficient one.
\end{abstract}