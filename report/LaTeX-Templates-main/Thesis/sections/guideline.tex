\section*{Introduction: about concepts used in the template}

In the template, we use a number of concepts that it is important to be clear about what they mean and how they relate to each other.
We illustrate this with examples.
You may have received your thesis as an assignment from a company, for example.
In this case, you will often have been given a problem that the company is experiencing and that you should try to find a solution to. 
In this case, the problem forms the background to the aim of the work and the question you are working on.

Example: company X has a system that they want to be able to use in a real-time application, but the performance of the system is unknown. 
The problem is then: the performance of the system is unknown.
The solution to the problem is to measure performance.
The purpose of your work will be to map the performance of the system so that you have a measure of this. 
The question can be formulated as: What is the performance of the system? The motivation for the work is that it is important to know the performance when the system is used for real-time applications.
Once you have the purpose and the question, you formulate the objectives that you want to achieve with the work, in this case the objectives could be to measure a number of different aspects of performance. 
Together, these objectives will then fulfil the purpose. 
However, your thesis does not have to be formulated as a specific problem to be solved.
Other examples of work that may appear as a thesis include:

\begin{itemize}
    \item[--] ``Case study'' or study of any phenomenon
    \item[--] Literature study
    \item[--] Investigate something, e.g. how users interact with a piece of software or how a design can be adapted to a particular group of users
    \item[--] Analysing e.g. comparing the performance of different software
    \item[--] Evaluate is often related to analysing something, your task may be to make a recommendation about which tool is best suited to a particular task
    \item[--] Explore new technologies or approaches. This may include developing an artefact, such as a piece of software or a system. 
    \item[--] Investigating an issue, e.g. by conducting a feasibility study
    \item[--] Developing and evaluating an algorithm, e.g. for a computational problem
    \end{itemize}
    
Of course, your thesis may also contain several of the above components. 
Common to all theses is that they must be thoroughly scientifically grounded, a thesis may not, for example, be merely an implementation.

In many of the examples above, there is no clearly specified problem to be solved. 
Instead, it may be a question to which you are seeking an answer, as in the example of evaluation.
However, all theses should have a purpose, question and motivation. 
The question should be framed in such a way that it can be answered in some way through the work you are doing.
But the answer can be abstract, for example, it can be to contribute to knowledge about the research question.
In the example where the task is to explore a technology, the question could be "What are the problems with developing XX"?
It is also common for the assignment to involve analysing and/or evaluating the artefact you have developed.
The question and the aim should be matched in such a way that when the aim is achieved, the question is answered.
In the text, we will use the term task for what you have to do, whether it is a problem to be solved or something else.
The text in the template is in English in this version, but for each heading the corresponding English words are in brackets.

This page should not be included in the final report.
\newpage